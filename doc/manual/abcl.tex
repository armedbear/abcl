% http://en.wikibooks.org/wiki/LaTeX/
% preamble for the Manual
%
% The goal is to move all the ``technical'' looking stuff here,
% leaving the manual itself as much as a pure content to be
% comfortably read and modified with a text editor.

\documentclass[10pt]{article}

\usepackage{color,hyperref}
\definecolor{darkblue}{rgb}{0.0,0.0,0.3}
\hypersetup{colorlinks,breaklinks,
            linkcolor=darkblue,urlcolor=darkblue,
            anchorcolor=darkblue,citecolor=darkblue}

\usepackage{a4wide}



\begin{document}
\title{A Manual for Armed Bear Common Lisp}
\date{August 4, 2011}
\author{Mark Evenson, Erik Huelsmann, Alessio Stallo, Ville Voutilainen}

\section{Introduction}

Armed Bear is a mostly conforming implementation of the ANSI Common
Lisp standard.  This manual documents the Armed Bear Common Lisp
implementation for users of the system.

\subsection{Version}
This manual corresponds to abcl-0.27.0, as yet unreleased.


\section{Interaction with host JVM}

% describe calling Java from Lisp, and calling Lisp from Java,
% probably in two separate sections.  Presumably, we can partition our
% audience into those who are more comfortable with Java, and those
% that are more comforable with Lisp

\subsection{Lisp to Java}

ABCL offers a number of mechanisms to manipulate Java libraries from
Lisp.

\begin{itemize}
\item Java values are accessible as objects of type JAVA:JAVA-OBJECT.
\item The Java FFI presents a Lisp package (JAVA) with many useful
  symbols for manipulating the artifacts of expectation on the JVM,
  including creation of new objects \ref{JAVA:JNEW}, \ref{JAVA:JMETHOD}), the
  introspection of values \ref{JAVA:JFIELD}, the execution of methods
  (\ref{JAVA:JCALL}, \ref{JAVA:JCALL-RAW}, \ref{JAVA:JSTATIC})
\item The JSS package (\ref{JSS}) in contrib introduces a convenient macro
  syntax \ref{JSS:SHARPSIGN_DOUBLEQUOTE_MACRO} for accessing Java
  methods, and additional convenience functions.
\item Java classes and libraries may be dynamically added to the
  classpath at runtime (JAVA:ADD-TO-CLASSPATH).
\end{itemize}

\subsection{Lisp from Java}

Manipulation of the Lisp API is currently lacking a stable interface,
so what is documented here is subject to change.  

\begin{itemize}
\item All Lisp values are descendants of LispObject.java
\item Lisp symbols are accessible via either directly referencing the
  Symbol.java instance or by dynamically introspecting the
  corresponding Package.java instance.
\item The Lisp dynamic environment may be saved via
  \code{LispThread.bindSpecial(Binding)} and restored via
  \code{LispThread.resetSpecialBindings(Mark)}.
\item Functions may be executed by invocation of the
  Function.execute(args [...]) 
\end{itemize}

\subsubsection{Lisp FFI}

FFI stands for "Foreign Function Interface" which is the phase which
the contemporary Lisp world refers to methods of "calling out" from
Lisp into "foreign" languages and environments.  This document
describes the various ways that one interacts with Lisp world of ABCL
from Java, considering the hosted Lisp as the "Foreign Function" that
needs to be "Interfaced".

\subsubsection{Calling Lisp from Java}

Note: As the entire ABCL Lisp system resides in the org.armedbear.lisp
package the following code snippets do not show the relevant import
statements in the interest of brevity.  An example of the import
statement would be

\lstset{language=Java}
\begin{lstlisting}
  import org.armedbear.lisp.*;
\end{lstliting}

to potentially import all the JVM symbol from the `org.armedbear.lisp'
namespace.

Per JVM, there can only ever be a single Lisp interpreter.  This is
started by calling the static method `Interpreter.createInstance()`.

\begin{code}[java]
  Interpreter interpreter = Interpreter.createInstance();
\end{code}

If this method has already been invoked in the lifetime of the current
Java process it will return null, so if you are writing Java whose
life-cycle is a bit out of your control (like in a Java servlet), a
safer invocation pattern might be:

\begin{code}[java]
  Interpreter interpreter = Interpreter.getInstance();
  if (interpreter == null) {
    interpreter = Interpreter.createInstance();
  }
\end{code}


The Lisp \code{eval} primitive may be simply passed strings for evaluation,
as follows

\begin{code}[java]
  String line = "(load \"file.lisp\")";
  LispObject result = interpreter.eval(line);
\end{code}

Notice that all possible return values from an arbitrary Lisp
computation are collapsed into a single return value.  Doing useful
further computation on the ``LispObject'' depends on knowing what the
result of the computation might be, usually involves some amount
of \code{instanceof} introspection, and forms a whole topic to itself
(c.f. [Introspecting a LispObject])

Using ``EVAL'' involves the Lisp interpreter.  Lisp functions may be
directly invoked by Java method calls as follows.  One simply locates
the package containing the symbol, then obtains a reference to the
symbol, and then invokes the `execute()` method with the desired
parameters.

\begin{code}[java]
    interpreter.eval("(defun foo (msg) (format nil \"You told me '~A'~%\" msg))");
    Package pkg = Packages.findPackage("CL-USER");
    Symbol foo = pkg.findAccessibleSymbol("FOO"); 
    Function fooFunction = (Function)foo.getSymbolFunction();
    JavaObject parameter = new JavaObject("Lisp is fun!");
    LispObject result = fooFunction.execute(parameter);
    // How to get the "naked string value"?
    System.out.println("The result was " + result.writeToString()); 
\end{code}

If one is calling an primitive function in the CL package the syntax
becomes considerably simpler if we can locate the instance of
definition in the ABCL source, we can invoke the symbol directly.  To
tell if a `LispObject` contains a reference to a symbol.

\begin{code}[java]
    boolean nullp(LispObject object) {
      LispObject result = Primitives.NULL.execute(object);
      if (result == NIL) {
        return false;
      }
      return true;
   }
\end{code}

\paragraph{Introspecting a LispObject}
\label{topic:Introspecting a LispObject}

We present various patterns for introspecting an an arbitrary
`LispObject` which can represent the result of every Lisp evaluation
into semantics that Java can meaniningfully deal with.

\paragraph{LispObject as \code{boolean}}

If the LispObject a generalized boolean values, one can use
\code{getBooleanValue()} to convert to Java:

\begin{code}[java]
     LispObject object = Symbol.NIL;
     boolean javaValue = object.getBooleanValue();
\end{code}

Although since in Lisp, any value other than NIL means "true", the
use of Java equality it quite a bit easier and more optimal:

\begin{code}[java]
    boolean javaValue = (object != Symbol.NIL);
\end{code}

\paragraph{LispObject is a list}

If LispObject is a list, it will have the type `Cons`.  One can then use
the \code{copyToArray} to make things a bit more suitable for Java
iteration.

\begin{code}[java]
    LispObject result = interpreter.eval("'(1 2 4 5)");
    if (result instanceof Cons) {
      LispObject array[] = ((Cons)result.copyToArray());
      ...
    }
\end{code}
    
A more Lispy way to iterated down a list is to use the `cdr()` access
function just as like one would traverse a list in Lisp:;

\begin{code}[java]
    LispObject result = interpreter.eval("'(1 2 4 5)");
    while (result != Symbol.NIL) {
      doSomething(result.car());
      result = result.cdr();
    }
\end{code}


\subsection{JAVA}

% include autogen docs for the JAVA package.

\section{ANSI Common Lisp Conformance}

ABCL is currently a non-conforming ANSI Common Lisp implementation due
to the following (known) issues:

\begin{itemize}
  \item Lack of long form of DEFINE-METHOD-COMBINATION
  \item Missing statement of conformance in accompanying documentation
  \item Incomplete MOP 
    % TODO go through AMOP with symbols, starting by looking for
    % matching function signature.
    % XXX is this really blocking ANSI conformance?  Answer: we have
    % to start with such a ``census'' to determine what we have.
\end{itemize}

ABCL aims to be be a fully conforming ANSI Common Lisp
implementation.  Any other behavior should be reported as a bug.

\section{Extensions}

The symbols in the EXTENSIONS package consititutes extensions to the
ANSI standard that are potentially useful to the user.  They include
functions for manipulating network sockets, running external programs,
registering object finalizers, constructing reference weakly held by
the garbage collector and others.

\paragraph{}
\label{EXTENSIONS:CADDR}
\index{CADDR}
--- Macro: \textbf{\%caddr} [\textbf{extensions}] \textit{}

\begin{adjustwidth}{5em}{5em}
not-documented
\end{adjustwidth}

\paragraph{}
\label{EXTENSIONS:CADR}
\index{CADR}
--- Macro: \textbf{\%cadr} [\textbf{extensions}] \textit{}

\begin{adjustwidth}{5em}{5em}
not-documented
\end{adjustwidth}

\paragraph{}
\label{EXTENSIONS:CAR}
\index{CAR}
--- Macro: \textbf{\%car} [\textbf{extensions}] \textit{}

\begin{adjustwidth}{5em}{5em}
not-documented
\end{adjustwidth}

\paragraph{}
\label{EXTENSIONS:CDR}
\index{CDR}
--- Macro: \textbf{\%cdr} [\textbf{extensions}] \textit{}

\begin{adjustwidth}{5em}{5em}
not-documented
\end{adjustwidth}

\paragraph{}
\label{EXTENSIONS:*AUTOLOAD-VERBOSE*}
\index{*AUTOLOAD-VERBOSE*}
--- Variable: \textbf{*autoload-verbose*} [\textbf{extensions}] \textit{}

\begin{adjustwidth}{5em}{5em}
not-documented
\end{adjustwidth}

\paragraph{}
\label{EXTENSIONS:*BATCH-MODE*}
\index{*BATCH-MODE*}
--- Variable: \textbf{*batch-mode*} [\textbf{extensions}] \textit{}

\begin{adjustwidth}{5em}{5em}
not-documented
\end{adjustwidth}

\paragraph{}
\label{EXTENSIONS:*COMMAND-LINE-ARGUMENT-LIST*}
\index{*COMMAND-LINE-ARGUMENT-LIST*}
--- Variable: \textbf{*command-line-argument-list*} [\textbf{extensions}] \textit{}

\begin{adjustwidth}{5em}{5em}
not-documented
\end{adjustwidth}

\paragraph{}
\label{EXTENSIONS:*DEBUG-CONDITION*}
\index{*DEBUG-CONDITION*}
--- Variable: \textbf{*debug-condition*} [\textbf{extensions}] \textit{}

\begin{adjustwidth}{5em}{5em}
not-documented
\end{adjustwidth}

\paragraph{}
\label{EXTENSIONS:*DEBUG-LEVEL*}
\index{*DEBUG-LEVEL*}
--- Variable: \textbf{*debug-level*} [\textbf{extensions}] \textit{}

\begin{adjustwidth}{5em}{5em}
not-documented
\end{adjustwidth}

\paragraph{}
\label{EXTENSIONS:*DISASSEMBLER*}
\index{*DISASSEMBLER*}
--- Variable: \textbf{*disassembler*} [\textbf{extensions}] \textit{}

\begin{adjustwidth}{5em}{5em}
not-documented
\end{adjustwidth}

\paragraph{}
\label{EXTENSIONS:*ED-FUNCTIONS*}
\index{*ED-FUNCTIONS*}
--- Variable: \textbf{*ed-functions*} [\textbf{extensions}] \textit{}

\begin{adjustwidth}{5em}{5em}
not-documented
\end{adjustwidth}

\paragraph{}
\label{EXTENSIONS:*ENABLE-INLINE-EXPANSION*}
\index{*ENABLE-INLINE-EXPANSION*}
--- Variable: \textbf{*enable-inline-expansion*} [\textbf{extensions}] \textit{}

\begin{adjustwidth}{5em}{5em}
not-documented
\end{adjustwidth}

\paragraph{}
\label{EXTENSIONS:*INSPECTOR-HOOK*}
\index{*INSPECTOR-HOOK*}
--- Variable: \textbf{*inspector-hook*} [\textbf{extensions}] \textit{}

\begin{adjustwidth}{5em}{5em}
not-documented
\end{adjustwidth}

\paragraph{}
\label{EXTENSIONS:*LISP-HOME*}
\index{*LISP-HOME*}
--- Variable: \textbf{*lisp-home*} [\textbf{extensions}] \textit{}

\begin{adjustwidth}{5em}{5em}
not-documented
\end{adjustwidth}

\paragraph{}
\label{EXTENSIONS:*LOAD-TRUENAME-FASL*}
\index{*LOAD-TRUENAME-FASL*}
--- Variable: \textbf{*load-truename-fasl*} [\textbf{extensions}] \textit{}

\begin{adjustwidth}{5em}{5em}
not-documented
\end{adjustwidth}

\paragraph{}
\label{EXTENSIONS:*PRINT-STRUCTURE*}
\index{*PRINT-STRUCTURE*}
--- Variable: \textbf{*print-structure*} [\textbf{extensions}] \textit{}

\begin{adjustwidth}{5em}{5em}
not-documented
\end{adjustwidth}

\paragraph{}
\label{EXTENSIONS:*REQUIRE-STACK-FRAME*}
\index{*REQUIRE-STACK-FRAME*}
--- Variable: \textbf{*require-stack-frame*} [\textbf{extensions}] \textit{}

\begin{adjustwidth}{5em}{5em}
not-documented
\end{adjustwidth}

\paragraph{}
\label{EXTENSIONS:*SAVED-BACKTRACE*}
\index{*SAVED-BACKTRACE*}
--- Variable: \textbf{*saved-backtrace*} [\textbf{extensions}] \textit{}

\begin{adjustwidth}{5em}{5em}
not-documented
\end{adjustwidth}

\paragraph{}
\label{EXTENSIONS:*SUPPRESS-COMPILER-WARNINGS*}
\index{*SUPPRESS-COMPILER-WARNINGS*}
--- Variable: \textbf{*suppress-compiler-warnings*} [\textbf{extensions}] \textit{}

\begin{adjustwidth}{5em}{5em}
not-documented
\end{adjustwidth}

\paragraph{}
\label{EXTENSIONS:*WARN-ON-REDEFINITION*}
\index{*WARN-ON-REDEFINITION*}
--- Variable: \textbf{*warn-on-redefinition*} [\textbf{extensions}] \textit{}

\begin{adjustwidth}{5em}{5em}
not-documented
\end{adjustwidth}

\paragraph{}
\label{EXTENSIONS:ADD-PACKAGE-LOCAL-NICKNAME}
\index{ADD-PACKAGE-LOCAL-NICKNAME}
--- Function: \textbf{add-package-local-nickname} [\textbf{extensions}] \textit{local-nickname actual-package \&optional (package-designator *package*)}

\begin{adjustwidth}{5em}{5em}
not-documented
\end{adjustwidth}

\paragraph{}
\label{EXTENSIONS:ADJOIN-EQL}
\index{ADJOIN-EQL}
--- Function: \textbf{adjoin-eql} [\textbf{extensions}] \textit{item list}

\begin{adjustwidth}{5em}{5em}
not-documented
\end{adjustwidth}

\paragraph{}
\label{EXTENSIONS:ARGLIST}
\index{ARGLIST}
--- Function: \textbf{arglist} [\textbf{extensions}] \textit{extended-function-designator}

\begin{adjustwidth}{5em}{5em}
not-documented
\end{adjustwidth}

\paragraph{}
\label{EXTENSIONS:ASSQ}
\index{ASSQ}
--- Function: \textbf{assq} [\textbf{extensions}] \textit{}

\begin{adjustwidth}{5em}{5em}
not-documented
\end{adjustwidth}

\paragraph{}
\label{EXTENSIONS:ASSQL}
\index{ASSQL}
--- Function: \textbf{assql} [\textbf{extensions}] \textit{}

\begin{adjustwidth}{5em}{5em}
not-documented
\end{adjustwidth}

\paragraph{}
\label{EXTENSIONS:AUTOLOAD}
\index{AUTOLOAD}
--- Function: \textbf{autoload} [\textbf{extensions}] \textit{symbol-or-symbols \&optional filename}

\begin{adjustwidth}{5em}{5em}
Setup the autoload for SYMBOL-OR-SYMBOLS optionally corresponding to FILENAME.
\end{adjustwidth}

\paragraph{}
\label{EXTENSIONS:AUTOLOAD-MACRO}
\index{AUTOLOAD-MACRO}
--- Function: \textbf{autoload-macro} [\textbf{extensions}] \textit{}

\begin{adjustwidth}{5em}{5em}
not-documented
\end{adjustwidth}

\paragraph{}
\label{EXTENSIONS:AUTOLOAD-REF-P}
\index{AUTOLOAD-REF-P}
--- Function: \textbf{autoload-ref-p} [\textbf{extensions}] \textit{symbol}

\begin{adjustwidth}{5em}{5em}
Boolean predicate for whether SYMBOL has generalized reference functions which need to be resolved.
\end{adjustwidth}

\paragraph{}
\label{EXTENSIONS:AUTOLOAD-SETF-EXPANDER}
\index{AUTOLOAD-SETF-EXPANDER}
--- Function: \textbf{autoload-setf-expander} [\textbf{extensions}] \textit{symbol-or-symbols filename}

\begin{adjustwidth}{5em}{5em}
Setup the autoload for SYMBOL-OR-SYMBOLS on the setf-expander from FILENAME.
\end{adjustwidth}

\paragraph{}
\label{EXTENSIONS:AUTOLOAD-SETF-FUNCTION}
\index{AUTOLOAD-SETF-FUNCTION}
--- Function: \textbf{autoload-setf-function} [\textbf{extensions}] \textit{symbol-or-symbols filename}

\begin{adjustwidth}{5em}{5em}
Setup the autoload for SYMBOL-OR-SYMBOLS on the setf-function from FILENAME.
\end{adjustwidth}

\paragraph{}
\label{EXTENSIONS:AUTOLOADP}
\index{AUTOLOADP}
--- Function: \textbf{autoloadp} [\textbf{extensions}] \textit{symbol}

\begin{adjustwidth}{5em}{5em}
Boolean predicate for whether SYMBOL stands for a function that currently needs to be autoloaded.
\end{adjustwidth}

\paragraph{}
\label{EXTENSIONS:CANCEL-FINALIZATION}
\index{CANCEL-FINALIZATION}
--- Function: \textbf{cancel-finalization} [\textbf{extensions}] \textit{object}

\begin{adjustwidth}{5em}{5em}
not-documented
\end{adjustwidth}

\paragraph{}
\label{EXTENSIONS:CHAR-TO-UTF8}
\index{CHAR-TO-UTF8}
--- Function: \textbf{char-to-utf8} [\textbf{extensions}] \textit{}

\begin{adjustwidth}{5em}{5em}
not-documented
\end{adjustwidth}

\paragraph{}
\label{EXTENSIONS:CHARPOS}
\index{CHARPOS}
--- Function: \textbf{charpos} [\textbf{extensions}] \textit{stream}

\begin{adjustwidth}{5em}{5em}
not-documented
\end{adjustwidth}

\paragraph{}
\label{EXTENSIONS:CLASSP}
\index{CLASSP}
--- Function: \textbf{classp} [\textbf{extensions}] \textit{}

\begin{adjustwidth}{5em}{5em}
Collect ({(Name [Initial-Value] [Function])}*) {Form}*
  Collect some values somehow.  Each of the collections specifies a bunch of
  things which collected during the evaluation of the body of the form.  The
  name of the collection is used to define a local macro, a la MACROLET.
  Within the body, this macro will evaluate each of its arguments and collect
  the result, returning the current value after the collection is done.  The
  body is evaluated as a PROGN; to get the final values when you are done, just
  call the collection macro with no arguments.

  Initial-Value is the value that the collection starts out with, which
  defaults to NIL.  Function is the function which does the collection.  It is
  a function which will accept two arguments: the value to be collected and the
  current collection.  The result of the function is made the new value for the
  collection.  As a totally magical special-case, the Function may be Collect,
  which tells us to build a list in forward order; this is the default.  If an
  Initial-Value is supplied for Collect, the stuff will be rplacd'd onto the
  end.  Note that Function may be anything that can appear in the functional
  position, including macros and lambdas.
\end{adjustwidth}

\paragraph{}
\label{EXTENSIONS:COLLECT}
\index{COLLECT}
--- Macro: \textbf{collect} [\textbf{extensions}] \textit{}

\begin{adjustwidth}{5em}{5em}
Collect ({(Name [Initial-Value] [Function])}*) {Form}*
  Collect some values somehow.  Each of the collections specifies a bunch of
  things which collected during the evaluation of the body of the form.  The
  name of the collection is used to define a local macro, a la MACROLET.
  Within the body, this macro will evaluate each of its arguments and collect
  the result, returning the current value after the collection is done.  The
  body is evaluated as a PROGN; to get the final values when you are done, just
  call the collection macro with no arguments.

  Initial-Value is the value that the collection starts out with, which
  defaults to NIL.  Function is the function which does the collection.  It is
  a function which will accept two arguments: the value to be collected and the
  current collection.  The result of the function is made the new value for the
  collection.  As a totally magical special-case, the Function may be Collect,
  which tells us to build a list in forward order; this is the default.  If an
  Initial-Value is supplied for Collect, the stuff will be rplacd'd onto the
  end.  Note that Function may be anything that can appear in the functional
  position, including macros and lambdas.
\end{adjustwidth}

\paragraph{}
\label{EXTENSIONS:COMPILE-SYSTEM}
\index{COMPILE-SYSTEM}
--- Function: \textbf{compile-system} [\textbf{extensions}] \textit{\&key quit (zip t) (cls-ext *compile-file-class-extension*) (abcl-ext *compile-file-type*) output-path}

\begin{adjustwidth}{5em}{5em}
not-documented
\end{adjustwidth}

\paragraph{}
\label{EXTENSIONS:DOUBLE-FLOAT-NEGATIVE-INFINITY}
\index{DOUBLE-FLOAT-NEGATIVE-INFINITY}
--- Variable: \textbf{double-float-negative-infinity} [\textbf{extensions}] \textit{}

\begin{adjustwidth}{5em}{5em}
not-documented
\end{adjustwidth}

\paragraph{}
\label{EXTENSIONS:DOUBLE-FLOAT-POSITIVE-INFINITY}
\index{DOUBLE-FLOAT-POSITIVE-INFINITY}
--- Variable: \textbf{double-float-positive-infinity} [\textbf{extensions}] \textit{}

\begin{adjustwidth}{5em}{5em}
not-documented
\end{adjustwidth}

\paragraph{}
\label{EXTENSIONS:DUMP-JAVA-STACK}
\index{DUMP-JAVA-STACK}
--- Function: \textbf{dump-java-stack} [\textbf{extensions}] \textit{}

\begin{adjustwidth}{5em}{5em}
not-documented
\end{adjustwidth}

\paragraph{}
\label{EXTENSIONS:EXIT}
\index{EXIT}
--- Function: \textbf{exit} [\textbf{extensions}] \textit{\&key status}

\begin{adjustwidth}{5em}{5em}
not-documented
\end{adjustwidth}

\paragraph{}
\label{EXTENSIONS:FEATUREP}
\index{FEATUREP}
--- Function: \textbf{featurep} [\textbf{extensions}] \textit{form}

\begin{adjustwidth}{5em}{5em}
not-documented
\end{adjustwidth}

\paragraph{}
\label{EXTENSIONS:FILE-DIRECTORY-P}
\index{FILE-DIRECTORY-P}
--- Function: \textbf{file-directory-p} [\textbf{extensions}] \textit{pathspec \&key (wild-error-p t)}

\begin{adjustwidth}{5em}{5em}
not-documented
\end{adjustwidth}

\paragraph{}
\label{EXTENSIONS:FINALIZE}
\index{FINALIZE}
--- Function: \textbf{finalize} [\textbf{extensions}] \textit{object function}

\begin{adjustwidth}{5em}{5em}
not-documented
\end{adjustwidth}

\paragraph{}
\label{EXTENSIONS:FIXNUMP}
\index{FIXNUMP}
--- Function: \textbf{fixnump} [\textbf{extensions}] \textit{}

\begin{adjustwidth}{5em}{5em}
not-documented
\end{adjustwidth}

\paragraph{}
\label{EXTENSIONS:GC}
\index{GC}
--- Function: \textbf{gc} [\textbf{extensions}] \textit{}

\begin{adjustwidth}{5em}{5em}
not-documented
\end{adjustwidth}

\paragraph{}
\label{EXTENSIONS:GET-FLOATING-POINT-MODES}
\index{GET-FLOATING-POINT-MODES}
--- Function: \textbf{get-floating-point-modes} [\textbf{extensions}] \textit{}

\begin{adjustwidth}{5em}{5em}
not-documented
\end{adjustwidth}

\paragraph{}
\label{EXTENSIONS:GET-PID}
\index{GET-PID}
--- Function: \textbf{get-pid} [\textbf{extensions}] \textit{}

\begin{adjustwidth}{5em}{5em}
Get the process identifier of this lisp process. 

Used to be in SLIME but generally useful, so now back in ABCL proper.
\end{adjustwidth}

\paragraph{}
\label{EXTENSIONS:GET-PID}
\index{GET-PID}
--- Function: \textbf{get-pid} [\textbf{extensions}] \textit{}

\begin{adjustwidth}{5em}{5em}
Get the process identifier of this lisp process. 

Used to be in SLIME but generally useful, so now back in ABCL proper.
\end{adjustwidth}

\paragraph{}
\label{EXTENSIONS:GET-SOCKET-STREAM}
\index{GET-SOCKET-STREAM}
--- Function: \textbf{get-socket-stream} [\textbf{extensions}] \textit{socket \&key (element-type (quote character)) (external-format default)}

\begin{adjustwidth}{5em}{5em}
:ELEMENT-TYPE must be CHARACTER or (UNSIGNED-BYTE 8); the default is CHARACTER.
EXTERNAL-FORMAT must be of the same format as specified for OPEN.
\end{adjustwidth}

\paragraph{}
\label{EXTENSIONS:GET-TIME-ZONE}
\index{GET-TIME-ZONE}
--- Function: \textbf{get-time-zone} [\textbf{extensions}] \textit{time-in-millis}

\begin{adjustwidth}{5em}{5em}
Return the timezone difference in hours for TIME-IN-MILLIS via the Daylight assumptions that were in place at its occurance. i.e. implement 'time of the time semantics'.
\end{adjustwidth}

\paragraph{}
\label{EXTENSIONS:GETENV}
\index{GETENV}
--- Function: \textbf{getenv} [\textbf{extensions}] \textit{variable}

\begin{adjustwidth}{5em}{5em}
Return the value of the environment VARIABLE if it exists, otherwise return NIL.
\end{adjustwidth}

\paragraph{}
\label{EXTENSIONS:GETENV-ALL}
\index{GETENV-ALL}
--- Function: \textbf{getenv-all} [\textbf{extensions}] \textit{variable}

\begin{adjustwidth}{5em}{5em}
Returns all environment variables as an alist containing (name . value)
\end{adjustwidth}

\paragraph{}
\label{EXTENSIONS:INIT-GUI}
\index{INIT-GUI}
--- Function: \textbf{init-gui} [\textbf{extensions}] \textit{}

\begin{adjustwidth}{5em}{5em}
not-documented
\end{adjustwidth}

\paragraph{}
\label{EXTENSIONS:INTERRUPT-LISP}
\index{INTERRUPT-LISP}
--- Function: \textbf{interrupt-lisp} [\textbf{extensions}] \textit{}

\begin{adjustwidth}{5em}{5em}
not-documented
\end{adjustwidth}

\paragraph{}
\label{EXTENSIONS:JAR-PATHNAME}
\index{JAR-PATHNAME}
--- Class: \textbf{jar-pathname} [\textbf{extensions}] \textit{}

\begin{adjustwidth}{5em}{5em}
not-documented
\end{adjustwidth}

\paragraph{}
\label{EXTENSIONS:MACROEXPAND-ALL}
\index{MACROEXPAND-ALL}
--- Function: \textbf{macroexpand-all} [\textbf{extensions}] \textit{form \&optional env}

\begin{adjustwidth}{5em}{5em}
not-documented
\end{adjustwidth}

\paragraph{}
\label{EXTENSIONS:MAILBOX}
\index{MAILBOX}
--- Class: \textbf{mailbox} [\textbf{extensions}] \textit{}

\begin{adjustwidth}{5em}{5em}
not-documented
\end{adjustwidth}

\paragraph{}
\label{EXTENSIONS:MAKE-DIALOG-PROMPT-STREAM}
\index{MAKE-DIALOG-PROMPT-STREAM}
--- Function: \textbf{make-dialog-prompt-stream} [\textbf{extensions}] \textit{}

\begin{adjustwidth}{5em}{5em}
not-documented
\end{adjustwidth}

\paragraph{}
\label{EXTENSIONS:MAKE-SERVER-SOCKET}
\index{MAKE-SERVER-SOCKET}
--- Function: \textbf{make-server-socket} [\textbf{extensions}] \textit{port}

\begin{adjustwidth}{5em}{5em}
Create a TCP server socket listening for clients on PORT.
\end{adjustwidth}

\paragraph{}
\label{EXTENSIONS:MAKE-SLIME-INPUT-STREAM}
\index{MAKE-SLIME-INPUT-STREAM}
--- Function: \textbf{make-slime-input-stream} [\textbf{extensions}] \textit{function output-stream}

\begin{adjustwidth}{5em}{5em}
not-documented
\end{adjustwidth}

\paragraph{}
\label{EXTENSIONS:MAKE-SLIME-OUTPUT-STREAM}
\index{MAKE-SLIME-OUTPUT-STREAM}
--- Function: \textbf{make-slime-output-stream} [\textbf{extensions}] \textit{function}

\begin{adjustwidth}{5em}{5em}
not-documented
\end{adjustwidth}

\paragraph{}
\label{EXTENSIONS:MAKE-SOCKET}
\index{MAKE-SOCKET}
--- Function: \textbf{make-socket} [\textbf{extensions}] \textit{host port}

\begin{adjustwidth}{5em}{5em}
Create a TCP socket for client communication to HOST on PORT.
\end{adjustwidth}

\paragraph{}
\label{EXTENSIONS:MAKE-TEMP-DIRECTORY}
\index{MAKE-TEMP-DIRECTORY}
--- Function: \textbf{make-temp-directory} [\textbf{extensions}] \textit{}

\begin{adjustwidth}{5em}{5em}
Create and return the pathname of a previously non-existent directory.
\end{adjustwidth}

\paragraph{}
\label{EXTENSIONS:MAKE-TEMP-FILE}
\index{MAKE-TEMP-FILE}
--- Function: \textbf{make-temp-file} [\textbf{extensions}] \textit{\&key prefix suffix}

\begin{adjustwidth}{5em}{5em}
Create and return the pathname of a previously non-existent file.
\end{adjustwidth}

\paragraph{}
\label{EXTENSIONS:MAKE-WEAK-REFERENCE}
\index{MAKE-WEAK-REFERENCE}
--- Function: \textbf{make-weak-reference} [\textbf{extensions}] \textit{obj}

\begin{adjustwidth}{5em}{5em}
not-documented
\end{adjustwidth}

\paragraph{}
\label{EXTENSIONS:MEMQ}
\index{MEMQ}
--- Function: \textbf{memq} [\textbf{extensions}] \textit{item list}

\begin{adjustwidth}{5em}{5em}
not-documented
\end{adjustwidth}

\paragraph{}
\label{EXTENSIONS:MEMQL}
\index{MEMQL}
--- Function: \textbf{memql} [\textbf{extensions}] \textit{item list}

\begin{adjustwidth}{5em}{5em}
not-documented
\end{adjustwidth}

\paragraph{}
\label{EXTENSIONS:MOST-NEGATIVE-JAVA-LONG}
\index{MOST-NEGATIVE-JAVA-LONG}
--- Variable: \textbf{most-negative-java-long} [\textbf{extensions}] \textit{}

\begin{adjustwidth}{5em}{5em}
not-documented
\end{adjustwidth}

\paragraph{}
\label{EXTENSIONS:MOST-POSITIVE-JAVA-LONG}
\index{MOST-POSITIVE-JAVA-LONG}
--- Variable: \textbf{most-positive-java-long} [\textbf{extensions}] \textit{}

\begin{adjustwidth}{5em}{5em}
not-documented
\end{adjustwidth}

\paragraph{}
\label{EXTENSIONS:MUTEX}
\index{MUTEX}
--- Class: \textbf{mutex} [\textbf{extensions}] \textit{}

\begin{adjustwidth}{5em}{5em}
not-documented
\end{adjustwidth}

\paragraph{}
\label{EXTENSIONS:NEQ}
\index{NEQ}
--- Function: \textbf{neq} [\textbf{extensions}] \textit{obj1 obj2}

\begin{adjustwidth}{5em}{5em}
not-documented
\end{adjustwidth}

\paragraph{}
\label{EXTENSIONS:NIL-VECTOR}
\index{NIL-VECTOR}
--- Class: \textbf{nil-vector} [\textbf{extensions}] \textit{}

\begin{adjustwidth}{5em}{5em}
not-documented
\end{adjustwidth}

\paragraph{}
\label{EXTENSIONS:OS-UNIX-P}
\index{OS-UNIX-P}
--- Function: \textbf{os-unix-p} [\textbf{extensions}] \textit{}

\begin{adjustwidth}{5em}{5em}
Is the underlying operating system some Unix variant?
\end{adjustwidth}

\paragraph{}
\label{EXTENSIONS:OS-WINDOWS-P}
\index{OS-WINDOWS-P}
--- Function: \textbf{os-windows-p} [\textbf{extensions}] \textit{}

\begin{adjustwidth}{5em}{5em}
Is the underlying operating system Microsoft Windows?
\end{adjustwidth}

\paragraph{}
\label{EXTENSIONS:PACKAGE-LOCAL-NICKNAMES}
\index{PACKAGE-LOCAL-NICKNAMES}
--- Function: \textbf{package-local-nicknames} [\textbf{extensions}] \textit{package}

\begin{adjustwidth}{5em}{5em}
not-documented
\end{adjustwidth}

\paragraph{}
\label{EXTENSIONS:PACKAGE-LOCALLY-NICKNAMED-BY-LIST}
\index{PACKAGE-LOCALLY-NICKNAMED-BY-LIST}
--- Function: \textbf{package-locally-nicknamed-by-list} [\textbf{extensions}] \textit{package}

\begin{adjustwidth}{5em}{5em}
not-documented
\end{adjustwidth}

\paragraph{}
\label{EXTENSIONS:PATHNAME-JAR-P}
\index{PATHNAME-JAR-P}
--- Function: \textbf{pathname-jar-p} [\textbf{extensions}] \textit{}

\begin{adjustwidth}{5em}{5em}
not-documented
\end{adjustwidth}

\paragraph{}
\label{EXTENSIONS:PATHNAME-URL-P}
\index{PATHNAME-URL-P}
--- Function: \textbf{pathname-url-p} [\textbf{extensions}] \textit{pathname}

\begin{adjustwidth}{5em}{5em}
Predicate for whether PATHNAME references a URL.
\end{adjustwidth}

\paragraph{}
\label{EXTENSIONS:PRECOMPILE}
\index{PRECOMPILE}
--- Function: \textbf{precompile} [\textbf{extensions}] \textit{name \&optional definition}

\begin{adjustwidth}{5em}{5em}
not-documented
\end{adjustwidth}

\paragraph{}
\label{EXTENSIONS:PROBE-DIRECTORY}
\index{PROBE-DIRECTORY}
--- Function: \textbf{probe-directory} [\textbf{extensions}] \textit{pathspec}

\begin{adjustwidth}{5em}{5em}
not-documented
\end{adjustwidth}

\paragraph{}
\label{EXTENSIONS:QUIT}
\index{QUIT}
--- Function: \textbf{quit} [\textbf{extensions}] \textit{\&key status}

\begin{adjustwidth}{5em}{5em}
not-documented
\end{adjustwidth}

\paragraph{}
\label{EXTENSIONS:READ-TIMEOUT}
\index{READ-TIMEOUT}
--- Function: \textbf{read-timeout} [\textbf{extensions}] \textit{socket seconds}

\begin{adjustwidth}{5em}{5em}
Time in SECONDS to set local implementation of 'SO\_RCVTIMEO' on SOCKET.
\end{adjustwidth}

\paragraph{}
\label{EXTENSIONS:REMOVE-PACKAGE-LOCAL-NICKNAME}
\index{REMOVE-PACKAGE-LOCAL-NICKNAME}
--- Function: \textbf{remove-package-local-nickname} [\textbf{extensions}] \textit{old-nickname \&optional package-designator}

\begin{adjustwidth}{5em}{5em}
not-documented
\end{adjustwidth}

\paragraph{}
\label{EXTENSIONS:RESOLVE}
\index{RESOLVE}
--- Function: \textbf{resolve} [\textbf{extensions}] \textit{symbol}

\begin{adjustwidth}{5em}{5em}
Resolve the function named by SYMBOL via the autoloader mechanism.
Returns either the function or NIL if no resolution was possible.
\end{adjustwidth}

\paragraph{}
\label{EXTENSIONS:RUN-SHELL-COMMAND}
\index{RUN-SHELL-COMMAND}
--- Function: \textbf{run-shell-command} [\textbf{extensions}] \textit{command \&key directory (output *standard-output*)}

\begin{adjustwidth}{5em}{5em}
not-documented
\end{adjustwidth}

\paragraph{}
\label{EXTENSIONS:SERVER-SOCKET-CLOSE}
\index{SERVER-SOCKET-CLOSE}
--- Function: \textbf{server-socket-close} [\textbf{extensions}] \textit{socket}

\begin{adjustwidth}{5em}{5em}
Close the server SOCKET.
\end{adjustwidth}

\paragraph{}
\label{EXTENSIONS:SET-FLOATING-POINT-MODES}
\index{SET-FLOATING-POINT-MODES}
--- Function: \textbf{set-floating-point-modes} [\textbf{extensions}] \textit{\&key traps}

\begin{adjustwidth}{5em}{5em}
not-documented
\end{adjustwidth}

\paragraph{}
\label{EXTENSIONS:SHOW-RESTARTS}
\index{SHOW-RESTARTS}
--- Function: \textbf{show-restarts} [\textbf{extensions}] \textit{restarts stream}

\begin{adjustwidth}{5em}{5em}
not-documented
\end{adjustwidth}

\paragraph{}
\label{EXTENSIONS:SIMPLE-STRING-FILL}
\index{SIMPLE-STRING-FILL}
--- Function: \textbf{simple-string-fill} [\textbf{extensions}] \textit{}

\begin{adjustwidth}{5em}{5em}
not-documented
\end{adjustwidth}

\paragraph{}
\label{EXTENSIONS:SIMPLE-STRING-SEARCH}
\index{SIMPLE-STRING-SEARCH}
--- Function: \textbf{simple-string-search} [\textbf{extensions}] \textit{}

\begin{adjustwidth}{5em}{5em}
not-documented
\end{adjustwidth}

\paragraph{}
\label{EXTENSIONS:SINGLE-FLOAT-NEGATIVE-INFINITY}
\index{SINGLE-FLOAT-NEGATIVE-INFINITY}
--- Variable: \textbf{single-float-negative-infinity} [\textbf{extensions}] \textit{}

\begin{adjustwidth}{5em}{5em}
not-documented
\end{adjustwidth}

\paragraph{}
\label{EXTENSIONS:SINGLE-FLOAT-POSITIVE-INFINITY}
\index{SINGLE-FLOAT-POSITIVE-INFINITY}
--- Variable: \textbf{single-float-positive-infinity} [\textbf{extensions}] \textit{}

\begin{adjustwidth}{5em}{5em}
not-documented
\end{adjustwidth}

\paragraph{}
\label{EXTENSIONS:SLIME-INPUT-STREAM}
\index{SLIME-INPUT-STREAM}
--- Class: \textbf{slime-input-stream} [\textbf{extensions}] \textit{}

\begin{adjustwidth}{5em}{5em}
not-documented
\end{adjustwidth}

\paragraph{}
\label{EXTENSIONS:SLIME-OUTPUT-STREAM}
\index{SLIME-OUTPUT-STREAM}
--- Class: \textbf{slime-output-stream} [\textbf{extensions}] \textit{}

\begin{adjustwidth}{5em}{5em}
not-documented
\end{adjustwidth}

\paragraph{}
\label{EXTENSIONS:SOCKET-ACCEPT}
\index{SOCKET-ACCEPT}
--- Function: \textbf{socket-accept} [\textbf{extensions}] \textit{socket}

\begin{adjustwidth}{5em}{5em}
Block until able to return a new socket for handling a incoming request to the specified server SOCKET.
\end{adjustwidth}

\paragraph{}
\label{EXTENSIONS:SOCKET-CLOSE}
\index{SOCKET-CLOSE}
--- Function: \textbf{socket-close} [\textbf{extensions}] \textit{socket}

\begin{adjustwidth}{5em}{5em}
Close the client SOCKET.
\end{adjustwidth}

\paragraph{}
\label{EXTENSIONS:SOCKET-LOCAL-ADDRESS}
\index{SOCKET-LOCAL-ADDRESS}
--- Function: \textbf{socket-local-address} [\textbf{extensions}] \textit{socket}

\begin{adjustwidth}{5em}{5em}
Returns the local address of the SOCKET as a dotted quad string.
\end{adjustwidth}

\paragraph{}
\label{EXTENSIONS:SOCKET-LOCAL-PORT}
\index{SOCKET-LOCAL-PORT}
--- Function: \textbf{socket-local-port} [\textbf{extensions}] \textit{socket}

\begin{adjustwidth}{5em}{5em}
Returns the local port number of the SOCKET.
\end{adjustwidth}

\paragraph{}
\label{EXTENSIONS:SOCKET-PEER-ADDRESS}
\index{SOCKET-PEER-ADDRESS}
--- Function: \textbf{socket-peer-address} [\textbf{extensions}] \textit{socket}

\begin{adjustwidth}{5em}{5em}
Returns the peer address of the SOCKET as a dotted quad string.
\end{adjustwidth}

\paragraph{}
\label{EXTENSIONS:SOCKET-PEER-PORT}
\index{SOCKET-PEER-PORT}
--- Function: \textbf{socket-peer-port} [\textbf{extensions}] \textit{socket}

\begin{adjustwidth}{5em}{5em}
Returns the peer port number of the given SOCKET.
\end{adjustwidth}

\paragraph{}
\label{EXTENSIONS:SOURCE}
\index{SOURCE}
--- Function: \textbf{source} [\textbf{extensions}] \textit{}

\begin{adjustwidth}{5em}{5em}
not-documented
\end{adjustwidth}

\paragraph{}
\label{EXTENSIONS:SOURCE-FILE-POSITION}
\index{SOURCE-FILE-POSITION}
--- Function: \textbf{source-file-position} [\textbf{extensions}] \textit{}

\begin{adjustwidth}{5em}{5em}
not-documented
\end{adjustwidth}

\paragraph{}
\label{EXTENSIONS:SOURCE-PATHNAME}
\index{SOURCE-PATHNAME}
--- Function: \textbf{source-pathname} [\textbf{extensions}] \textit{symbol}

\begin{adjustwidth}{5em}{5em}
Returns either the pathname corresponding to the file from which this symbol was compiled,or the keyword :TOP-LEVEL.
\end{adjustwidth}

\paragraph{}
\label{EXTENSIONS:SPECIAL-VARIABLE-P}
\index{SPECIAL-VARIABLE-P}
--- Function: \textbf{special-variable-p} [\textbf{extensions}] \textit{}

\begin{adjustwidth}{5em}{5em}
not-documented
\end{adjustwidth}

\paragraph{}
\label{EXTENSIONS:STRING-FIND}
\index{STRING-FIND}
--- Function: \textbf{string-find} [\textbf{extensions}] \textit{char string}

\begin{adjustwidth}{5em}{5em}
not-documented
\end{adjustwidth}

\paragraph{}
\label{EXTENSIONS:STRING-INPUT-STREAM-CURRENT}
\index{STRING-INPUT-STREAM-CURRENT}
--- Function: \textbf{string-input-stream-current} [\textbf{extensions}] \textit{stream}

\begin{adjustwidth}{5em}{5em}
not-documented
\end{adjustwidth}

\paragraph{}
\label{EXTENSIONS:STRING-POSITION}
\index{STRING-POSITION}
--- Function: \textbf{string-position} [\textbf{extensions}] \textit{}

\begin{adjustwidth}{5em}{5em}
not-documented
\end{adjustwidth}

\paragraph{}
\label{EXTENSIONS:STYLE-WARN}
\index{STYLE-WARN}
--- Function: \textbf{style-warn} [\textbf{extensions}] \textit{format-control \&rest format-arguments}

\begin{adjustwidth}{5em}{5em}
not-documented
\end{adjustwidth}

\paragraph{}
\label{EXTENSIONS:TRULY-THE}
\index{TRULY-THE}
--- Macro: \textbf{truly-the} [\textbf{extensions}] \textit{}

\begin{adjustwidth}{5em}{5em}
not-documented
\end{adjustwidth}

\paragraph{}
\label{EXTENSIONS:UPTIME}
\index{UPTIME}
--- Function: \textbf{uptime} [\textbf{extensions}] \textit{}

\begin{adjustwidth}{5em}{5em}
not-documented
\end{adjustwidth}

\paragraph{}
\label{EXTENSIONS:URI-DECODE}
\index{URI-DECODE}
--- Function: \textbf{uri-decode} [\textbf{extensions}] \textit{}

\begin{adjustwidth}{5em}{5em}
not-documented
\end{adjustwidth}

\paragraph{}
\label{EXTENSIONS:URI-ENCODE}
\index{URI-ENCODE}
--- Function: \textbf{uri-encode} [\textbf{extensions}] \textit{}

\begin{adjustwidth}{5em}{5em}
not-documented
\end{adjustwidth}

\paragraph{}
\label{EXTENSIONS:URL-PATHNAME}
\index{URL-PATHNAME}
--- Class: \textbf{url-pathname} [\textbf{extensions}] \textit{}

\begin{adjustwidth}{5em}{5em}
not-documented
\end{adjustwidth}

\paragraph{}
\label{EXTENSIONS:URL-PATHNAME-AUTHORITY}
\index{URL-PATHNAME-AUTHORITY}
--- Function: \textbf{url-pathname-authority} [\textbf{extensions}] \textit{p}

\begin{adjustwidth}{5em}{5em}
not-documented
\end{adjustwidth}

\paragraph{}
\label{EXTENSIONS:URL-PATHNAME-FRAGMENT}
\index{URL-PATHNAME-FRAGMENT}
--- Function: \textbf{url-pathname-fragment} [\textbf{extensions}] \textit{p}

\begin{adjustwidth}{5em}{5em}
not-documented
\end{adjustwidth}

\paragraph{}
\label{EXTENSIONS:URL-PATHNAME-QUERY}
\index{URL-PATHNAME-QUERY}
--- Function: \textbf{url-pathname-query} [\textbf{extensions}] \textit{p}

\begin{adjustwidth}{5em}{5em}
not-documented
\end{adjustwidth}

\paragraph{}
\label{EXTENSIONS:URL-PATHNAME-SCHEME}
\index{URL-PATHNAME-SCHEME}
--- Function: \textbf{url-pathname-scheme} [\textbf{extensions}] \textit{p}

\begin{adjustwidth}{5em}{5em}
not-documented
\end{adjustwidth}

\paragraph{}
\label{EXTENSIONS:WEAK-REFERENCE}
\index{WEAK-REFERENCE}
--- Class: \textbf{weak-reference} [\textbf{extensions}] \textit{}

\begin{adjustwidth}{5em}{5em}
not-documented
\end{adjustwidth}

\paragraph{}
\label{EXTENSIONS:WEAK-REFERENCE-VALUE}
\index{WEAK-REFERENCE-VALUE}
--- Function: \textbf{weak-reference-value} [\textbf{extensions}] \textit{obj}

\begin{adjustwidth}{5em}{5em}
Returns two values, the first being the value of the weak ref,the second T if the reference is valid, or NIL if it hasbeen cleared.
\end{adjustwidth}

\paragraph{}
\label{EXTENSIONS:WRITE-TIMEOUT}
\index{WRITE-TIMEOUT}
--- Function: \textbf{write-timeout} [\textbf{extensions}] \textit{socket seconds}

\begin{adjustwidth}{5em}{5em}
No-op setting of write timeout to SECONDS on SOCKET.
\end{adjustwidth}



\subsection{Beyond ANSI}

Naturally, in striving to be a useful contemporary Common Lisp
implementation, ABCL endeavors to include extensions beyond the ANSI
specification which are either widely adopted or are especially useful
in working with the hosting JVM.

\subsubsection{Extensions to CLOS}

There is an additional syntax for specializing the parameter of a
generic function on a java class, viz. \code{(java:jclass CLASS-STRING)}
where \code{CLASS-STRING} is a string naming a Java class in dotted package
form.

For instance the following specialization would perhaps allow one to
print more information about the contents of a java.util.Collection
object

\begin{code}[lisp]
(defmethod print-object ((coll (java:jclass "java.util.Collection")) stream)
\ldots
\end{code}

If the class had been loaded via a classloader other than the original
the class you wish to specialize on, one needs to specify the
classloader as an optional third argument.

\begin{code}[lisp]
(defmethod print-object ((device-id (java:jclass "dto.nbi.service.hdm.alcatel.com.nNBIDeviceID" 
                                    (\#"getBaseLoader" cl-user::*classpath-manager*)))
\ldots
\end{code}

\subsubsection{Extensions to the Reader}

We implement a special hexadecimal escape sequence for specifying
characters to the Lisp reader, namely we allow a sequences of the form
\# \textbackslash Uxxxx to be processed by the reader as character whose code is
specified by the hexadecimal digits ``xxxx''.  The hexadecimal sequence
must be exactly four digits long, padded by leading zeros for values
less than 0x1000.

Note that this sequence is never output by the implementation.  Instead,
the corresponding Unicode character is output for characters whose
code is greater than 0x00ff.

\section{Multithreading}

% TODO document the THREADS package.
\begin{verbatim}
THREADS:CURRENT-THREAD
  Function: (not documented)
THREADS:DESTROY-THREAD
  Function: (not documented)
THREADS:GET-MUTEX
  Function: Acquires a lock on the `mutex'.
THREADS:INTERRUPT-THREAD
  Function: Interrupts THREAD and forces it to apply FUNCTION to ARGS.
THREADS:MAILBOX-EMPTY-P
  Function: Returns non-NIL if the mailbox can be read from, NIL otherwise.
THREADS:MAILBOX-PEEK
  Function: Returns two values. The second returns non-NIL when the mailbox
THREADS:MAILBOX-READ
  Function: Blocks on the mailbox until an item is available for reading.
THREADS:MAILBOX-SEND
  Function: Sends an item into the mailbox, notifying 1 waiter
THREADS:MAKE-MAILBOX
  Function: (not documented)
THREADS:MAKE-MUTEX
  Function: (not documented)
THREADS:MAKE-THREAD
  Function: (not documented)
THREADS:MAKE-THREAD-LOCK
  Function: Returns an object to be used with the `with-thread-lock' macro.
THREADS:MAPCAR-THREADS
  Function: (not documented)
THREADS:OBJECT-NOTIFY
  Function: (not documented)
THREADS:OBJECT-NOTIFY-ALL
  Function: (not documented)
THREADS:OBJECT-WAIT
  Function: (not documented)
THREADS:RELEASE-MUTEX
  Function: Releases a lock on the `mutex'.
THREADS:SYNCHRONIZED-ON
  Function: (not documented)
THREADS:THREAD
  Class: (not documented)
THREADS:THREAD-ALIVE-P
  Function: Boolean predicate whether THREAD is alive.
THREADS:THREAD-JOIN
  Function: Waits for thread to finish.
THREADS:THREAD-NAME
  Function: (not documented)
THREADS:THREADP
  Function: (not documented)
THREADS:WITH-MUTEX
  Function: (not documented)
THREADS:WITH-THREAD-LOCK
  Function: (not documented)
\end{verbatim}


\section{History}

ABCL was originally the extension language for the J editor, which was
started in 1998 by Peter Graves.  Sometime in 2003, it seems that a
lot of code that had previously not been released publically was
suddenly committed that enabled ABCL to be plausibly termed an ANSI
Common Lisp implementation.  

In 2006, the implementation was transferred to the current
maintainers, who have strived to improve its usability as a
contemporary Common Lisp implementation.


\end{document}

% TODO
%   1.  Create mechanism for swigging DocString and Lisp docs into
%       sections.

